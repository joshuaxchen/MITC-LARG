\documentclass{standalone}

% graphics
\usepackage[usenames,dvipsnames]{xcolor}
\usepackage{tikz}
\usetikzlibrary{calc}
\usepackage{pgfplots}
\usepackage{siunitx}
\usetikzlibrary{patterns}

\begin{document}


\begin{tikzpicture}
\pgfplotsset{
      every x tick label/.append style={font=\Large},
      every y tick label/.append style={font=\Large}
  }

 \begin{axis}[
        ybar,
        bar width=15pt,
	width=1.9\textwidth,
	nodes near coords,
        % (changed from `xtick`)
        xtick distance=1,
        ylabel={\huge Outflow},
        xlabel={\huge Main inflow}, % (changed to an absolute value)
	x label style={at={(axis description cs:0.5,-0.03)},anchor=north},
	y label style={at={(axis description cs:-0.030,0.6)}, anchor=south},
        %enlarge x limits={abs=0.5},
	enlarge y limits={value=0.2,upper},
        % ---------------------------------------------------------------------
        % changes to get what you want
        % ---------------------------------------------------------------------
	xtick={1600, 1700, 1800, 1900, 2000},
        scaled ticks=false,
	xmin=1550,
	xmax=2050,
	legend pos=north east,
	point meta=explicit symbolic,
        % remove the `xticks`
        xtick style={
            /pgfplots/major tick length=0pt,
        },
%        % `ybar interval` is not what you want here I guess, because then e.g.
%        % the last bar is not drawn. To simulate that ...
%        % ... add `extra x ticks` between the `xticks` ...
%        extra x ticks={1.5,...,8.5},
%        % ... without stating any label ...
%        extra x tick labels={},
%        % ... and show the major gridlines
%        extra x tick style={
%            xmajorgrids,
%        },
    ]

	\addplot+ [
		black,
		fill=white,
		pattern color=.,
            error bars/.cd,
                %y dir=both,
                % (changed from `y explicit` so the error bars are (clearly)
                % visible
                y explicit,
       ] coordinates {
	(1600,1577.34)+-(0, 29.66)
	(1700,1578.02)+-(0, 34.83)
	(1800,1566.94)+-(0, 23.02)
	(1900,1567.22)+-(0, 21.26)
	(2000,1563.30)+-(0, 16.07)
        };\legend{human-baseline}

	\addplot+ [
		%pattern=grid,
		%pattern color=red,
            error bars/.cd,
                %y dir=both,
                % (changed from `y explicit` so the error bars are (clearly)
                % visible
                y explicit,
        ] coordinates {
	(1600,1588.14) +- (0, 30.60) 
	(1700,1587.13) +- (0, 33.11)
	(1800,1578.71) +- (0, 29.24)
	(1900,1574.32) +- (0, 20.31)
	(2000,1572.41) +- (0, 19.98)
        };\legend{AVP1} 

	
	\addplot+ [
		%pattern=dots,
		%pattern color=brown,
            error bars/.cd,
                %y dir=both,
                % (changed from `y explicit` so the error bars are (clearly)
                % visible
                y explicit,
       ] coordinates {
	(1600,1597.07)+-(0,30.24)
	(1700,1595.66)+-(0,35.27)
	(1800,1586.63)+-(0,27.48)
	(1900,1587.24)+-(0,24.18)
	(2000,1582.13)+-(0,24.09)
        };\legend{AVP2}

	\addplot+ [
	    error bars/.cd,
                %y dir=both,
                % (changed from `y explicit` so the error bars are (clearly)
                % visible
                y explicit,
       ] coordinates {
	(1600,1610.32)+-(0, 30.33)
	(1700,1599.8 )+-(0, 31.85)
	(1800,1593.47)+-(0, 24.68)
	(1900,1592.46)+-(0, 28.49)
	(2000,1591.6 )+-(0, 26.05)
        };\legend{AVP3}

	\addplot+ [
		%pattern=crosshatch,
		%pattern color=purple,
            error bars/.cd,
                %y dir=both,
                % (changed from `y explicit` so the error bars are (clearly)
                % visible
                y explicit,
       ] coordinates {
	(1600,1608.80)+-(0, 34.48)
	(1700,1609.78)+-(0, 37.35)
	(1800,1602.54)+-(0, 28.50)
	(1900,1598.76)+-(0, 27.40)
	(2000,1600.56)+-(0, 25.34)
        };\legend{AVP4}

	\addplot+ [
		%pattern=crosshatch dots,%fivepointed stars,
		%pattern color=green,
            error bars/.cd,
                %y dir=both,
                % (changed from `y explicit` so the error bars are (clearly)
                % visible
                y explicit,
       ] coordinates {
	(1600,1616.33)+-(0, 41.77)
	(1700,1614.53)+-(0, 36.66)
	(1800,1607.33)+-(0, 30.46)
	(1900,1604.48)+-(0, 29.11)
	(2000,1609.85)+-(0, 29.94)
        };\legend{AVP5}


	\legend{human-baseline, AVP=1\%, AVP=2\%, AVP=3\%, AVP=4\%, AVP=5\%}

	\coordinate (A11) at (30,26);
	\coordinate (A12) at (43,35);
	\coordinate (A13) at (57,48);
	\coordinate (A14) at (71,47);
	\coordinate (A15) at (85,55);

	\coordinate (A21) at (130,26);
	\coordinate (A22) at (143,34);
	\coordinate (A23) at (157,38);
	\coordinate (A24) at (171,48);
	\coordinate (A25) at (185,53);

	\coordinate (A31) at (230,17);
	\coordinate (A32) at (243,25);
	\coordinate (A33) at (257,32);
	\coordinate (A34) at (271,41);
	\coordinate (A35) at (285,46);

	\coordinate (A41) at (330,13);
	\coordinate (A42) at (343,25);
	\coordinate (A43) at (357,31);
	\coordinate (A44) at (371,37);
	\coordinate (A45) at (385,43);

	\coordinate (A51) at (430,11);
	\coordinate (A52) at (443,21);
	\coordinate (A53) at (457,30);
	\coordinate (A54) at (471,39);
	\coordinate (A55) at (485,48);

        %\legend{
        %    x data,
        %    y values,
        %}
    \end{axis}

\node at (A11) {\huge *};
\node at (A12) {\huge *};
\node at (A13) {\huge *};
\node at (A14) {\huge *};
\node at (A15) {\huge *};

\node at (A21) {\huge *};
\node at (A22) {\huge *};
\node at (A23) {\huge *};
\node at (A24) {\huge *};
\node at (A25) {\huge *};

\node at (A31) {\huge *};
\node at (A32) {\huge *};
\node at (A33) {\huge *};
\node at (A34) {\huge *};
\node at (A35) {\huge *};

\node at (A41) {\huge *};
\node at (A42) {\huge *};
\node at (A43) {\huge *};
\node at (A44) {\huge *};
\node at (A45) {\huge *};

%\node at (A51) {\huge *};
\node at (A52) {\huge *};
\node at (A53) {\huge *};
\node at (A54) {\huge *};
\node at (A55) {\huge *};

\node[above,font=\huge, align=left] at (current bounding box.north) {Training: random vehicle placement, main inflow 2000, AVP=30\%,\\ Evaluation:
random vehicle placement, main inflow [1600,1800], AVP=[0,5\%]};
\end{tikzpicture}
\end{document}
